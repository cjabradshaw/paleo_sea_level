\section{Summary of results}

This is a summary of the results of the modelling. There are a total of six models with which are compared. In addition, these tables give how many sea level indicators, number of marine limiting, number of terrestrial limiting, and number of sea level index points.

The sea level is calculated at the location of each data point. To evaluate how well the calculated curve fits the data point, a score is assigned. This metric was originally used by \citet{GowanEtal2016}. The score is the discrepancy, in number of meters, the calculated sea level falls outside of the constraint plus the error bars. A score is zero if the calculated sea level is consistent with the data point. As an example, if the calculated sea level curve is below a terrestrial limiting point, it is given a score of zero. The sum of the scores for each location for each model are shown in the tables. A warning about the scores is that a lower score does not necessarily mean a better fit, as it will depend on the age distribution of the indicators, and the number of indicators of a specific kind. For example, if there are a lot of marine limiting data points, a calculated curve that is over a hundred meters above those indicators may provide a good score, but it is not necessarily a good fit. As a result, it is a good idea to also look at the plotted curves for visual inspection. \citet{GowanEtal2016} found a good result by normallizing the scores based on the number of the number of each type of indicator at each location, but we have not done that here.

\import{statistics_tex/}{summary.tex}
