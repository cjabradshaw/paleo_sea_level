\documentclass[a4paper,12pt]{article}

\usepackage[margin=2cm]{geometry}
\usepackage{graphicx}
\usepackage{tabularx}
\usepackage{supertabular}
\usepackage{float}
\usepackage{multirow}
\usepackage{import}
\usepackage[sort]{natbib}
\usepackage{times}
\usepackage[hidelinks]{hyperref}

\usepackage[nottoc,notlot,notlof]{tocbibind}

\usepackage{fontspec}
\setmainfont{FreeSerif}
\setsansfont{FreeSans}
\setmonofont{FreeMono}

\usepackage{polyglossia}
\setdefaultlanguage{english}
\setotherlanguages{russian}

% Japanese text
\usepackage{zxjatype}
\setjamainfont{ipaexm.ttf}

\setlength{\parindent}{0pt}
\setlength{\parskip}{1em}

\begin{document}


\title{Comparision of calculated and measured paleo-sea level using different lower mantle viscosity values and PaleoMIST 1.0 \\
\normalsize As a supplement to ``\emph{A new global ice sheet reconstruction for the past 80\,000 years}" by Evan J. Gowan, Xu Zhang, Sara Khosravi, Alessio Rovere, Paolo Stocchi, Anna L. C. Hughes, Richard Gyllencreutz, Jan Mangerud, John-Inge Svendsen \& Gerrit Lohmann}

\author{Evan James Gowan}
    
\date{}

\maketitle

\tableofcontents

\newpage

\section{Purpose of this document}

In this report there is a detailed summary, including plots, of a worldwide compilation of paleo-sea level data, and six ice sheet-Earth models. In this particular report, we compare the standard version of PaleoMIST 1.0 (with 2500 year time steps and using a lower mantle viscosity of $4\times10^{22}$ Pa~s), with five other Earth models with viscosity values ranging between $10^{21}$ and $10^{23}$. When developing PaleoMIST 1.0, a variety of lower mantle viscosity values were tested, and it was found that a value approaching $10^{23}$ Pa~s provided the best trade-off between increasing the amount of ice in the center of the Laurentide Ice Sheet and fitting the sea level data. This ended up being true for the Eurasian ice sheets as well. PaleoMIST 1.0 was tuned to an Earth model with a viscosity of $4\times10^{22}$ Pa~s, but the comparison shown in this document demonstrate that a slightly higher value of $10^{23}$ Pa~s provides an even better fit.

The accompanying paper is \citet{GowanEtal2021b}.

Update on October 22, 2021:

This document has been updated to include several additional sites at the LGM and MIS 3. It also has fixed an error in the Cairns and Mackay sites caused by incorrectly subtracting half of the depth range rather than adding it. I apologize for this error. For the coral data for Tahiti and Huon Peninsula, it was originally set to be marine limiting, since the living range was tens of meters. We now use the 2-sigma range determined by \citet{HibbertEtal2016}. We include the interpretations of sea level range by \citet{IshiwaEtal2019} and \citet{YokoyamaEtal2000} for the Bonaparte Gulf shallow marine/estuary/intertidal data in addition to my conservative marine limiting assignment. I also included the interpreted sea level of Huon Peninsula by \citet{deGelderEtal2021} for MIS 3 to compare with the coral depth range interpretation by \citet{HibbertEtal2016}. Finally, I also recalibrated all the radiocarbon dates using updated calibration curves published in 2020 \citep{HeatonEtal2020,HoggEtal2020,ReimerEtal2020}.

Update on March 14, 2021:

I have included data from the Baltic Sea and North Sea.

Update on July 4, 2021:

In this update, data from Antarctica are included. I have also updated the figures so that index points are now drawn as rectangles, rather than the green dots as before. I have used different shades of green depending on whether or not the indicator uncertainty is below or above 10 m.


\section{Summary of ice and Earth models}


In order to make the figures compact, I have made shorthand codes for the ice and Earth models. I calculate each ice sheet separately, and the numbers refer to the ``run number", which is a sequential number that I used to distinguish git commits (see \url{https://github.com/evangowan/icesheet}). The ice model numbering scheme is as follows:

``North America"\_``Europe"\_``Antarctica"\_``Patagonia"

For PaleoMIST 1.0, the minimal MIS~3 configuration reconstruction is 72\_73\_74\_75, while the maximal configuration is 82\_83\_85\_85

For the Earth models, I created a shorthand scheme during my PHD, which I have continued to use. A full explanation can be found on the github page:

\url{https://github.com/evangowan/icesheet/blob/master/global/earth_model_format_codes.txt}

The full description of each model compared in this document is in this section.



\subsection{Ice models}

\import{temp/}{ice_models.tex}

\subsection{Earth models}

\import{temp/}{earth_models.tex}

\newpage


\import{./}{compilations.tex}

\newpage

\import{./}{statistics.tex}


\newpage

\import{figure_tex/}{summary.tex}

\clearpage

\newpage

% bibliography
\bibliographystyle{copernicus}
\bibliography{references.bib}

\end{document}
