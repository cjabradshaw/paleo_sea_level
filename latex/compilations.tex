\section{Paleo-sea level compilations}

This is a list of paleo-sea level compilations, which served as the basis for this report. We acknowledge the hard work of the people compiling the data, as well as acknowledging those who collected the original data.

\subsection{North America}

\begin{itemize}
  \item Canada and Greenland - A.S. Dyke and T.S. James (unpublished)
  \item Eastern Canada - \citet{VacchiEtal2018}
  \item Hudson Bay - \citet{SimonEtal2016}
  \item Hudson Bay and northern mainland Canada - \citet{GowanEtal2016}
\end{itemize}

I have made some changes and corrections from the compilations above.

At Churchill, there is a site, denoted with the radiocarbon date S-738, which was originally assigned to be a marine limiting indicator. It was described in \citet{MorlanEtal2000} as "shells enclosed in gravel in a quartzite ridge". It was originally interpreted as being a sea level indicator, with sea level at around 35 m. Using IMCalc  \citep{LorscheidRovere2019}, and a tidal amplitude of 1.6 m based on the tide gauge at Churchill \citep{Ray2016}, assuming the landform represents a beach ridge, and including a 20\% uncertainty on the original 35 m elevation (to account for the lack of information on elevation measurement), the sea level indicator is 32.8$\pm$7 m.


\subsection{Europe}

The Baltic Sea sea level indicators is from an unpublished compilation provided by Holger Steffen:

Rosentau et al. (in prep.) A Holocene relative sea-level database for the Baltic Sea.

When the paper becomes available, I will add the appropriate references.

Scandinavia and Svalbard sea level indicators are from and unpublished compilation by Jan Mangerud, Kristian Vasskog and \O{}ystein Lohne. Though I have not fully referenced anything in the compilation, awaiting the full publication, some parts of the compilation can be found in:

\begin{itemize}
  \item Svalbard - \citet{BondevikEtal1995}
  \item Northern Europe - \citet{FormanEtal2004}
  \item Norway - \citet{VasskogEtal2019}
\end{itemize}

