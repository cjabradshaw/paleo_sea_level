\section{Paleo-sea level compilations}

This is a list of paleo-sea level compilations, which served as the basis for this report. We acknowledge the hard work of the people compiling the data, as well as acknowledging those who collected the original data.

\subsection{North America}

\begin{itemize}
  \item Canada and Greenland - A.S. Dyke and T.S. James (unpublished, though some of it was summarized in \citet{DykePeltier2000})
  \item Eastern Canada - \citet{VacchiEtal2018}
  \item Hudson Bay - \citet{SimonEtal2016}
  \item Hudson Bay and northern mainland Canada - \citet{GowanEtal2016}
\end{itemize}

I have made some changes and corrections from the compilations above.

At Churchill, there is a site, denoted with the radiocarbon date S-738, which was originally assigned to be a marine limiting indicator. It was described in \citet{MorlanEtal2000} as "shells enclosed in gravel in a quartzite ridge". It was originally interpreted as being a sea level indicator, with sea level at around 35 m. Using IMCalc  \citep{LorscheidRovere2019}, and a tidal amplitude of 1.6 m based on the tide gauge at Churchill \citep{Ray2016}, assuming the landform represents a beach ridge, and including a 20\% uncertainty on the original 35 m elevation (to account for the lack of information on elevation measurement), the sea level indicator is 32.8$\pm$7 m.

There were many data that refered just to compilations rather than the original sources. I have tried to track down the original sources as much as possible, but in some cases it was not possible, as they were neither listed in the Vacchi compilation nor the Dyke and James compilation.

The compilation of sea level indicators in the eastern United States was done by \citet{EngelhartHorton2012}. Thanks to Simon Engelhart for sending me a copy of the dataset with the reservoir corrections used for marine organisms.

The MIS 3-5 data from the east coast of the United States was compiled by \citet{PicoEtal2017}.

\subsection{Europe}

\label{sec:Europe}

The Baltic Sea sea level indicators are from \citep{RosentauEtal2021}. Note that some of the regions that they designated were really large with the gradient of the GIA, so I made smaller regions. This is why the regions in this report do not correspond to theirs in many places. Also note that Rosentau \emph{et al} chose to enter the radiocarbon dates for {\AA}ngermanland as pre-calibrated dates. I have not changed them.


Scandinavia sea level indicators are from and unpublished compilation by Jan Mangerud, Kristian Vasskog and \O{}ystein Lohne. Some parts of the compilation can be found in:

\begin{itemize}
  \item Svalbard - \citet{BondevikEtal1995}
  \item Northern Europe - \citet{FormanEtal2004}
  \item Norway - \citet{LohneEtal2007,RomundsetEtal2010,RomundsetEtal2011,RomundsetEtal2015,RomundsetEtal2018,VasskogEtal2019}
\end{itemize}

The compilation of sea level indicators for Rotterdam in the Netherlands is from \citet{HijmaCohen2019}.


\subsection{Eurasian Arctic}

The sea level indicators for northern Norway and Svalbard are from and unpublished compilation by Jan Mangerud, Kristian Vasskog and \O{}ystein Lohne (see details in Section~\ref{sec:Europe}).

The compilation of sea level indicators for northern Russia comes from \citet{BaranskayaEtal2018}. Thank you to Alisa V. Baranskaya for sending the references (including translations from Russian) that were missing from the published compilation.

\subsection{Southeastern Asia}

The sea level indicators from southeastern Asia were compiled by \citet{MannEtal2019}.

\subsection{Tropical Corals}

Corals from tropical regions were compiled by \citet{HibbertEtal2016}. In this report, we have taken indicators for Huon Peninsula, Vanuatu and French Polynesia from this database.

